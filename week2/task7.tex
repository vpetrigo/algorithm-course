Задача:
Докажите, что если частоты всех символов меньше 1/3 (другими словами, каждый символ в исходную строку s входит строго
меньше |s|/3 раз), то коды всех символов в коде Хаффмана будут длиннее одного бита.

Доказательство:
Задача:
Докажите, что если частоты всех символов меньше 1/3 (другими словами, каждый символ в исходную строку s входит строго
меньше |s|/3 раз), то коды всех символов в коде Хаффмана будут длиннее одного бита.

Доказательство:
Для начала рассмотрим пример. Пусть у нас есть строка $abcd$. Найдем частоту вхождений всех символов:
$
1. \: a: \: 1 \\
2. \: b: \: 1 \\
3. \: c: \: 1 \\
4. \: d: \: 1
$
То есть частоты всех символов меньше $\frac{1}{3}$:
$f_i = \frac{1}{4} \le \frac{1}{3}$

Построенное дерево будет иметь вид:
       root
    /        \
   2          2
 /   \      /   \
a     b    c     d

Видно, что в данном случае для каждого символа получится 2-битный код.
Обобщим полученное:
Так как для построения дерева Хаффмана у нас исползуется упорядоченный список частот символов
$\frac{1}{3} > f_1 \geq f_2 \geq \ldots \geq f_m$
и на каждом шаге мы складываем две минимальные величины, формируя новое число f'=f_{m-1} + f_m и помещается обратно в
список.
Если частота какого-то символа объединилась с какой-то другой, то из способа построения следует, что его кодовое слово
будет имеет не менее 2-бит. То есть, если какое-то кодовое слово получится однобитным, то соответствующая ему частота
оставалась в списке неизменной до окончания процесса. $\blacksquare$