$\textbf{Задача:}$
Докажите, что для любого целого $n \geq 0$ числа $F_n \; и \; F_{n + 1}$ взаимно просты (то есть их наибольший общий
делитель равен 1).
$\textbf{Доказательство:}$
$\textbf{База:}$
$n = 3 \: \colon \: F_3 = 1, \; F_4 = 2. \; GCD(F_3, F_4) = 1$
$\textbf{Переход:}$
Пускай для некоторого $m \leq n$ условие, что $GCD(F_m, F_{m + 1}) = 1$
Докажем, что и для $n + 1$ данное утверждение будет верно:
$GCD(F_n, F_{n + 1}) = GCD(F_n, F_n + F_{n - 1}) = GCD(F_n, F_{n - 1}) = 1 \quad \blacksquare$

Переход от $GCD(F_n, F_n + F_{n - 1})$ к $GCD(F_n, F_{n - 1})$ был произведен исходя из факта, что число
$F_n + F_{n - 1} > F_n$ и $(F_n + F_{n - 1}) \mod F_n \equiv F_{n - 1}$