Задача:
Постройте алгоритм, который по данному массиву A[1…n] выводит его минимальные sqrt(n) элементов в порядке возрастания (другими
словами, выводит A′[1…sqrt(n)]) за время O(n).

Доказательство:
Поставленную задачу решим с помощью сортировки кучей.
------------------------------------
Процедура $\texttt{GetSqrtMinElem(A[1...n])}$:
$\texttt{BuildMinHeap(A)}$
для $i$ от $1$ до $\sqrt{n}$:
    $\texttt{ExtractMin(A)}$
    $\texttt{SiftDown(A, 1)}$
------------------------------------

В процедуре используются следующие функции и допущения:
1. $\texttt{ExtractMin(A)}$ - извлекает минимальный элемент, на его место ставит последний листовой элемент
2. $\texttt{BuildMaxHeap(A)}$ - построение $\max$-кучи
3. Считаем, что $\sqrt{n}$ вычисляется за $O(1)$ или известен заранее

Оценка на время работы:
Процедура $\texttt{BuildMinHeap(A)}$ строит $\max$-кучу за время $O(n)$, процедура просеивания вниз $\texttt{SiftDown(A, 1)}$ занимает время $O(\log n)$. $\texttt{ExtractMin(A)}$ и вычисление $\sqrt{n}$ занимают константное время работы $O(1)$. Так как в процессе работы необходимо для $\sqrt{n}$-элементов выполнить процедуру просеивания вниз, то можем сделать заключение об общем времени работы алгоритма:
$T(\text{общее}) \leq O(n) + O(\sqrt{n}\log{n}) + O(1) = O(n) \quad\blacksquare$