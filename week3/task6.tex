Задача:
Предположим, что в нашем распоряжении есть алгоритм, который за линейное время находит медиану массива. Алгоритм не
изменяет массив и выдает индекс ячейки исходного массива, в которой стоит медиана. Покажите, как использовать этот алгоритм,
чтобы за линейное же время найти любую заданную порядковую статистику массива.

Доказательство:
Так как мы можем за линейное время найти медиану и её индекс в массиве $A[i]$, то далее используя процедуру разбиения на
части мы можем сделать следующее:
1. Разделим исходный массив на 3 части: $|A_l|$, $|A_c = A'\left[\left\lfloor\frac{n}{2}\right\rfloor\right]$, $|A_r|$.
 Причем заметим, что размеры подмассивов $|A_l|$ и $|A_r|$ $\displaystyle{\leq \frac{n}{2}}$, так как разбиваемся относительно медианы.
2. После этого сравниваем число $k$ (требуемой порядковой статистики с полученными границами подмассивов).
Если $k = A_c$, то $k$-порядковая статистика найдена и равна медиане, если нет, то мы переходим в одну из частей ($|A_l|$
или $|A_r|$).

Используя выкладки из предыдущих пунктов получаем следующее рекуррентное соотношение: $T(n) = O(n) + T(\frac{n}{2}) \Rightarrow
T(n) = O(n) \qquad \blacksquare$